\documentclass[a4paper]{report}

\usepackage[margin=2.5cm]{geometry} % set margins
\usepackage{amsmath} % for align
\usepackage{url}
\usepackage{hyperref}
\usepackage{natbib} % citation styling
\usepackage[dvipsnames]{xcolor} % text colors
\usepackage[utf8]{inputenc}
\usepackage{multirow} % merged rows for table
\usepackage[toc]{glossaries}
\usepackage[toc]{appendix} % appendix



% TODO: FONT SELECTION

% CODE FORMATTING
\usepackage{listings}
\lstset{
    basicstyle=\ttfamily\footnotesize,
    breaklines=true,
    frame=single,
    rulecolor=\color{gray},
    backgroundcolor=\color{white},
    numbers=left,
    numberstyle=\tiny\color{gray},
    keywordstyle=\color{blue},
    commentstyle=\color{green!50!black}
}

% HEADING FORMATTING
\usepackage{titlesec}
\titleformat{\chapter}[display]
{\normalfont\bfseries\Large}{\chaptertitlename\ \thechapter}{20pt}{\Huge}
\titlespacing*{\chapter}{0pt}{-30pt}{20pt}

% PARAGRAPH FORMATTING
\usepackage{parskip}
\setlength{\parskip}{0.5em}
\usepackage{setspace}
\onehalfspacing

% HYPERREFERENCES
\hypersetup{
    colorlinks=true,
    linkcolor=blue,
    filecolor=magenta,
    urlcolor=cyan,
}



%% Glossary
\makeglossaries
% SHA-2 Terminology
\newglossaryentry{sha2}{
    name={Secure Hashing Algorithm 2 (SHA-2)},
    description={DEFINE ME!}
}
\newglossaryentry{hash function}{
  name={Hash Function},
  description={A cryptographic algorithm that deterministically maps arbitrary-length input data to a fixed-size hash digest, ensuring properties like collision resistance, preimage resistance, and computational efficiency for verifying data integrity and authenticity}
}
\newglossaryentry{hash digest}{
  name={Hash Digest},
  description={Also known as simply "hash", is a fixed-size output produced by a hash function}
}
\newglossaryentry{compression}{
  name={Compression},
  description={A function that combines the current chaining vector and a message block to produce the next state}
}
\newglossaryentry{expansion}{
  name={Expansion},
  description={Preprocessing step where the message block is expanded into a schedule of words for use in hash computation rounds}
}
\newglossaryentry{truncation}{
    name={Truncation},
    description={The process of shortening the final hash digest to a specified bit-length, as defined by the \cite{secure_hash_standard}}
}
\newglossaryentry{message}{
  name={Message},
  description={Input data processed by the hash function, padded and divided into fixed-size blocks for hashing, as defined by the \cite{secure_hash_standard}}
}

% Vector Types
\newglossaryentry{IV}{
    name={Initial Vector (IV)},
    description={Predefined initial constants, based on hash function used, to initialize the algorithm's state before processing the input message}
}
\newglossaryentry{CV}{
    name={Chaining Vector (CV)},
    description={Intermediate state values created during message expansion, and used as input for processing each given block iteratively}
}

% Collition Types
\newglossaryentry{FS}{
    name={Free-start collision (FS)},
    description={A free-start collision involves finding two messages, either distinct or identical, that produce identical hash digests, where each message utilies its own distinct chosen IV}
}
\newglossaryentry{SFS}{
    name={Semi-free-start collision (SFS)},
    description={A semi-free-start collision involved finding two distinct messages that produce identical hash digests under a chosen IV}
}
\newglossaryentry{STD}{
    name={Standard collision (STD)},
    description={A standard collision involves finding two distinct messages that produce identical hash digests under a fixed initial value. This is the classic collision resistance security property required of cryptographic hash functions}
}

% Cryptanalysis Terms
\newglossaryentry{collision}{
  name={Collision},
  description={A security vulnerability where two distinct inputs produce the same hash digest}
}

\newglossaryentry{differential}{
  name={Differential},
  description={Controlled differences in input messages analyzed to trace propagation through hash rounds}
}

% SMT Terms
\newglossaryentry{smt}{
  name={SMT},
  description={A Satisfiability Modulo Theory (SMT) solver is a tool that determines the satisfiability of logical formulas with respect to combinations of background theories}
}



\title{\textbf{Improving SHA-2 Collisions Using Satisfiability Modulo Theory (SMT) Solvers}}
\author{Barlik Marcel \\ City St George's University of London \\ marcel.barlik@citystgeorges.ac.uk}



\begin{document}
\maketitle


\section{Abstract}
\textcolor{red}{TODO!}


\tableofcontents
\setcounter{tocdepth}{3}


\section{Introduction}
\paragraph{
    This research paper investigates potential measurable quantifiable performance differences in \gls{smt} solvers and their parameters.
    Ideally, this helps shape future research with the knowledge and understanding of impacts certain solving lemmas make on \gls{sha2} collisions.
}

\paragraph{
    \textcolor{orange}{Utilise unexplored opportunities that have arisen from advances in \cite{y_et_al_2024}. This research will expand on the novel concept of using a \gls{smt} solver for practical \gls{collision}s, using the principles and mathematics described, in addition to their code as a reference.}
}

\subsection{Research Questions}
\begin{center}
  \begin{enumerate}
      \item Does using a more effective SMT solver yield better SHA-256 collision results?
    \item Can encodings provided in the research by \cite{y_et_al_2024} be improved upon, aiming for better practical SHA-256 collisions?
  \end{enumerate}
\end{center}


\section{Codebase}
\subsection{Information}
\paragraph{
    All code is written in Rust (rustc version 1.85.1), compiled with the provided LLVM backend, and linked with mold (version 2.37.1). \cite{rust}, \cite{mold}
}

\paragraph{
    The code is licensed under CC BY-NC-SA 4.0, allowing free use with attribution for further research, while limiting any commercial use. \cite{cc_by-nc-sa}
}

\paragraph{
    Repository for the code can be found here, with further build information in the README.md file:
}

\begin{center}
    \url{https://github.com/Supermarcel10/CSG-IN3007}
\end{center}


\subsection{Functionality}
\paragraph{
    The functionality of the software is split into multiple parts, working together to form a working package:
}

\begin{itemize}
    \item SMTLIB Generation
    \item Benchmarking
    \item Verification
    \item Graph Plotting
\end{itemize}


\section{Methodology}
\subsection{Benchmarking}
In order to obtain quantifiable results, benchmarks were used in order to obtain time and memory information.


An X86\_64 machine was set up for this, with no background tasks or workers, and an otherwise fresh installation of linux.
Each run was ran sequentially, as to ensure no performance degradation, with a 15 minute timeout, and stop tloerance of 3.
The parameters, hardware and software were used in this specific configuraiton, for all figures presented, unless otherwise specified.

The full runner specification:
\begin{center}
    \begin{tabular}{|r|c|}
        \hline
        & \textbf{Runner Specification} \\
        \hline
        \textbf{CPU} & AMD Ryzen 9 5900X \\
        \textbf{MEM} & 4 x 32GiB DDR4 3600MHz CL 36 \\
        \textbf{OS} & NixOS 25.05 (Warbler) x86\_64 \\
        \textbf{KERNEL} & Linux Realtime 6.6.77-rt50 \\
        \hline
    \end{tabular}
\end{center}

\begin{center}
    \begin{tabular}{|r|c|}
        \hline
        \textbf{Solver} & \textbf{Version} \\
        \hline
        Z3 & 4.13.4 \\
        CVC5 & 1.2.1 \\
        Yices & 2.6.5 \\
        Bitwuzla & 0.7.0 \\
        Boolector & 3.2.3 \\
        STP & 2.3.4 \\
        \hline
    \end{tabular}
\end{center}
% TODO: Add footnote about Bitwuzla issue
% TODO: Add references to all of these tools


\section{Findings}
Summary of collision attacks on SHA-2.
Time in seconds, memory in MiB.

\begin{center}
\begin{tabular}{c c c c c c c c}
    Hash Function & Attack Type & Collision Type & Rounds & Time & Memory & References & Year \\
    \hline \hline
    \multirow{3}{*}{SHA-256}
                  & Differential & STD & 31 & $2^{49.8}$ & $2^{48}$ & \cite{y_et_al_2024} & 2024 \\
                  & Brute Force & STD & 18 & $2^{5.5}$ & $2^{6.6}$ & - & 2025 \\
                  & Differential & SFS & 39 & practical & - & \cite{y_et_al_2024} & 2024 \\
    \hline
    \multirow{3}{*}{SHA-512}
        & Differential & STD & 28 & practical & - & \cite{y_et_al_2024} & 2024 \\
        & Differential & STD & 31 & $2^{115.6}$ & $2^{77.3}$ & \cite{y_et_al_2024} & 2024 \\
        & Brute Force & STD & 18 & $2^{5.4}$ & $2^{7.4}$ & - & 2025 \\
\end{tabular}
\end{center}


% Glossary
\printglossaries
% \glsaddall


% References
% TODO: Harvard-style formatting
\bibliographystyle{plainnat}
\bibliography{references}


\begin{appendices}
\section{Notes}
This report was written in TeX and compiled to PDF.
\end{appendices}

\section{Report TODO}
- Talk about SMT solver choice
- Talk about how other solvers had other issues.
- Talk about Bitwuzla issues (3)
- Talk about figures
- Talk about settings used to generate figures
- Talk about results
- Talk about the code
- Talk about benchmark methodology
- Talk about results

\end{document}
